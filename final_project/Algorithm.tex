\section{Algorithm}
\subsection{Introduction to cross entropy algorithm}
\paragraph{Motivation}
Cross entropy is already applied to solve some complicate optimization problems, such as travelling salesman problem(TSP), and  the max-cut problem. On the other hand, cross entropy can also be used to perform rare event-simulation. Consider the property of our problem, we can figure out that if we want to determine a certain node should be chosen or not to be chosen base on the probability indicate by the algorithm output, this output must be fairly approach to 0 or 1. And this property is 
consist of the cross entropy in rare-event simulation. Hence we would like to use cross entropy algorithm to find the optimization solution of our sensor selection problem. 
\paragraph{Brief description}
To apply the cross entropy algorithm. The constraint and objective must be specified, which will be 
propose in the following subsection. After that, the CE method involves an iterative procedure where each iteration can be broken down into two phases.

\begin{enumerate}
\item Generate a random data sample (trajectories, vectors, etc.) according to a specified mechanism.
\item Update the parameters of the random mechanism based on the data to produce “better” sample 
in the next iteration.
\end{enumerate}  
After iteration which is large enough, we can make the parameter as our optimization solution.

\subsection{Set up}
Here, given the channel capacity constraint and the objective function. First, the channel 
\subsection{Finding the optimization solution}


