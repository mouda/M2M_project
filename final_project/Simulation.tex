\section{Model of Video Streaming}
\subsection{Video Distortion Model}
In \cite{distortion}, it introduces this video distortion model.
For this concurrent video streaming application, video packets are
transmitted over the wireless mesh network and need to meet their
playout deadline. Then the decoded video distortion, $D_{dec}$, is
given by:

\begin{equation} D^{dec} = D_{enc} + D_{loss} \end{equation}
\begin{equation} D_{enc} = D_{0} + \theta/(R-R_{0}) \end{equation}
\begin{equation} D_{loss} = k(P_{v} + [1-P_{v}](1-P(T_v>\delta_v))) \end{equation}

. In \cite{distortion}, the parameters of the model are
$D_0=0.49$, $\theta=1222$, $R_0=10.39kbps$, $k=185$.

\subsection{Network Model}
Before we introduce the network model, we should define some
parameters first \cite{belair,GA,fixed}. Consider a WMN with N
nodes distributed on a plane with $c_{ij}$ as the distance between
two nodes $i$ and $j$. Each wireless mesh node is equipped with a
radio having transmission range $r_i$ and interference range
$r'_i$. Hence, we can model such a WMN by a directed graph
$\mathbf{G=G(N,L)}$, where $\mathbf{N}$ represents the set of
nodes in the WMN, and $\mathbf{L}$ represents the set of wireless
directed links between nodes. A direct link $\lbrace ij \rbrace$
exists from node $i$ to node $j$ if $c_{ij} \le r_{i}$ and $i \ne
j$. We characterize link $\lbrace ij \rbrace \in \mathbf{L}$
using:

Besides, we let the path of video clip $v$ be $L_v$, which
includes a set of link $\lbrace ij \rbrace$. For each link
$\lbrace ij \rbrace$ we define an index variable:

\begin{equation}
X_{ij}^v = \left\{\begin{array}{ll}
  1,  & \lbrace ij \rbrace \in L_v\\
  0,  & \lbrace ij \rbrace \notin L_v
\end{array} \right. \forall \lbrace ij \rbrace \in \mathbf{L}, \forall v \in
\mathbf{V}
\end{equation}

\subsubsection{Packet loss rate and delay}
Assume that the bit error rate (BER) on link $\lbrace ij \rbrace$
is $e_{ij}$, and S is the average packet size. Then the end-to-end
packet loss rate associated with path $L_v$ can be defined as:
\begin{equation} P_v = 1 -  \prod_{{\lbrace ij
\rbrace \in \mathbf{L}}} (1- X ^{v}_{ij}), \forall v \in
\mathbf{V}
\end{equation}

$P_{ij} = 1-(1-e_{ij})^S$ is the packet loss on link $\lbrace ij
\rbrace$.

\subsubsection{Interference model}
To model the delay, we assume the delay $t_{ij}$ on link $\lbrace
ij \rbrace$ is an exponential distribution with the probability
density function $f_{t_{ij}}(x) \sim \lambda e^{-\lambda_{ij}x}$,
$\forall \lbrace ij \rbrace \in \mathbf{L}$. As a result, the
end-to-end delay for path $L_v$ can be derived as:

When the number of hops along $L_v$ is not large enough, we use
the moment generating function, $G_{T_v(s)}$, to find the
probability that the packet streamed over path $L_v$ has delivery
delay longer than $\delta_v$. Then we can get this probability,
$P(T_v>\delta_v)$, when the number of hops is small as
\cite{error}:

\begin{eqnarray}
P(T_v>\delta_v)&=&\int_{\delta_v}^{\infty} \, f_{T_v}(x)\, dx \nonumber\\
               &=&\sum_{\lbrace ij \rbrace \in L_v} \lbrace
(\lambda_{ij}-X_{ij}^vs)G_{T_v(s)}\rbrace \mid_{s=\lambda_{ij}} \nonumber\\
\end{eqnarray}

\subsubsection{Aggregated traffic}
In this part, we want to model the total traffic on the link
$\lbrace ij \rbrace$. And we define $L_v^i$ as the sub-path which
includes all the links along the path $L_v$ up to the link
$\lbrace ij \rbrace$. Therefore, we can derived the total traffic
on link $\lbrace ij \rbrace$, $\rho_{ij}$, as:

\begin{equation} \rho_{ij} = \sum_{v \in \mathbf{V}} \prod_{{\lbrace
kn \rbrace \in L_v^i}} \lbrace(1- P_{kn})X ^{v}_{kn}\rbrace \times
R_v , \forall \lbrace ij \rbrace \in \mathbf{L} \end{equation}

%\subsubsection{Interference model}
\subsection{Route selection} Moreover, the aggregate traffic
on each link $\lbrace ij \rbrace$ should be guaranteed that it can
not exceed its available bandwidth $R_{ij}$. Hence, the problem of
routing loop also should be prevented. After concluding the above
constraints, we can formulate the optimal route selection problem
as: given a WMN $\mathbf{G(N,L)}$ and a set of video clips, for
$\mathbf{V}$ streaming requests, find a set of paths so that the
aggregate distortion of $\mathbf{V}$ concurrent video sessions is
minimized. And this route selection problem can be mathematic
formulated as:

\begin{itemize}
\item Minimize:
\begin{equation}
\sum_{v \in \mathbf{V)}} D_v
\end{equation}
\item Subject to:

\begin{equation}
\rho_{ij} \le R_{ij}, \forall \lbrace ij \rbrace \in \mathbf{L}
\end{equation}

\begin{eqnarray}
\sum_{j:\lbrace ij \rbrace \in L_v} X_{ij}^v
\left\{\begin{array}{ll}
  =0,    & i: i \in L_v, L_v^i = \Phi\\
  \le1,  & otherwise
\end{array} \right.,\nonumber\\
\forall i \in \mathbf{N}, \forall v \in \mathbf{V}
\end{eqnarray}

\end{itemize}

where $D_v$ is the average distortion of received video clip $v$
at the receive node $\gamma_v$. (10) is the rate constraint to
satisfy. (11), (12), and (13) guarantee each path $L_v$ provides a
loop-free connection.